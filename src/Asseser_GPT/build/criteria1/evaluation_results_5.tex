\documentclass[11pt]{article}
\usepackage[utf8]{inputenc}
\usepackage{fullpage}
\usepackage{helvet}
\renewcommand{\rmdefault}{phv} % Use Helvetica
\usepackage{array}

\title{Assessment of the CRC Proposal for the EASE Project}
\author{}
\date{}

\begin{document}
\maketitle

\section{Assessment Criteria and Ratings}

\begin{table}[h]
    \centering
    \begin{tabular}{| m{5cm} | m{2cm} | m{7cm} |}
        \hline
        \textbf{Criterion} & \textbf{Rating (1-10)} & \textbf{Comments} \\
        \hline
        Clarity & 4 & The writing has technical jargon that could confuse non-expert readers. The use of acronyms is not consistently explained, reducing accessibility. \newline \textbf{Suggestions:} Define acronyms upon first use, simplify complex sentences. \\
        \hline
        Precision & 5 & While some methodologies are sufficiently detailed, several statements are vague. \newline \textbf{Suggestions:} Include specifics on methods and results. Reduce ambiguous terms. \\
        \hline
        Coherence and Flow & 6 & The overall structure is logical, but transitions between sections could be smoother. \newline \textbf{Suggestions:} Use transitional phrases to guide the reader through the text. Review sections to minimize abrupt shifts. \\
        \hline
        Professional Tone & 7 & The tone is largely professional; however, some informal language is present. \newline \textbf{Suggestions:} Remove colloquial phrases and ensure consistency in the formal tone. \\
        \hline
        Engagement and Persuasiveness & 6 & The proposal has potential but lacks compelling arguments in some sections. \newline \textbf{Suggestions:} Highlight the unique aspects of the research more effectively to engage the reader. \\
        \hline
        Grammar, Syntax, and Formatting & 5 & There are several grammatical errors, and some formatting issues need to be addressed. \newline \textbf{Suggestions:} Proofread for grammatical correctness and ensure formatting is consistent throughout the document. \\
        \hline
    \end{tabular}
    \caption{Summary Ratings and Key Findings}
\end{table}

\section{Strengths}
\begin{itemize}
    \item Clear mathematical representations and equations.
    \item Logical organization of technical content.
    \item Professional terminology used appropriately in some areas.
\end{itemize}

\section{Weaknesses}
\begin{itemize}
    \item Technical jargon may alienate non-expert readers.
    \item Lack of clear explanations for acronyms and specialized terms.
    \item Some vague statements that lack precision.
    \item Occasional use of informal language.
    \item Grammatical and formatting inconsistencies.
\end{itemize}

\section{Recommendations}
\begin{itemize}
    \item Define all acronyms and technical terms on first usage for clarity.
    \item Provide additional context for methodologies to enhance precision.
    \item Improve transitions between sections to enhance coherence.
    \item Maintain a consistent professional tone throughout the document.
    \item Highlight unique contributions and societal impacts more prominently.
    \item Conduct thorough proofreading and formatting checks before final submission.
\end{itemize}

\end{document}