\documentclass[11pt]{article}
\usepackage{fullpage}
\usepackage{helvet}
\renewcommand\rmdefault{phv}
\usepackage{amsmath}
\usepackage{array}


\title{Assessment of the Collaborative Research Centre (CRC) Proposal for the EASE Project}
\author{}
\date{}

\begin{document}
\maketitle

\section{Assessment of Writing Style}

\subsection{Clarity}
\begin{itemize}
    \item **Rating**: 5/10
    \item **Strengths**:
        \begin{itemize}
            \item Technical concepts, like Bayes theorem, are centrally located and define relationships clearly.
        \end{itemize}
    \item **Weaknesses**:
        \begin{itemize}
            \item Some sentences are overly complex and could be simplified for better understanding.
            \item Acronyms are not consistently explained when first used.
        \end{itemize}
    \item **Recommendations**:
        \begin{itemize}
            \item Break down complex sentences for enhanced readability.
            \item Ensure that all acronyms are defined at their first mention.
        \end{itemize}
\end{itemize}

\subsection{Precision}
\begin{itemize}
    \item **Rating**: 6/10
    \item **Strengths**:
        \begin{itemize}
            \item Use of formal notation helps convey precise mathematical relationships.
        \end{itemize}
    \item **Weaknesses**:
        \begin{itemize}
            \item Some statements are vague, such as this assumption is also called without explicitly stating what the assumption is.
        \end{itemize}
    \item **Recommendations**:
        \begin{itemize}
            \item Avoid vague phrases; replace them with direct statements that specify definitions.
            \item Ensure all components of equations are adequately described.
        \end{itemize}
\end{itemize}

\subsection{Coherence and Flow}
\begin{itemize}
    \item **Rating**: 4/10
    \item **Strengths**:
        \begin{itemize}
            \item The proposal contains logical sections divided by topics.
        \end{itemize}
    \item **Weaknesses**:
        \begin{itemize}
            \item Transitions between ideas and sections are abrupt and could disrupt reading flow.
            \item Repetition of ideas (e.g., posterior distribution) without clear elaboration.
        \end{itemize}
    \item **Recommendations**:
        \begin{itemize}
            \item Add transition sentences to link sections.
            \item Aim to introduce each concept gradually to maintain narrative flow.
        \end{itemize}
\end{itemize}

\subsection{Professional Tone}
\begin{itemize}
    \item **Rating**: 7/10
    \item **Strengths**:
        \begin{itemize}
            \item The language used aligns well with academic and scientific standards.
        \end{itemize}
    \item **Weaknesses**:
        \begin{itemize}
            \item Occasional use of informal phrasing like hopelessly infeasible detracts from the overall tone.
        \end{itemize}
    \item **Recommendations**:
        \begin{itemize}
            \item Ensure all language is formal; replace informal phrases with appropriate academic language.
        \end{itemize}
\end{itemize}

\subsection{Engagement and Persuasiveness}
\begin{itemize}
    \item **Rating**: 5/10
    \item **Strengths**:
        \begin{itemize}
            \item Illustrates the significance of the research, such as applications of Bayesian networks.
        \end{itemize}
    \item **Weaknesses**:
        \begin{itemize}
            \item Fails to create a strong compelling narrative that gets the reader invested.
        \end{itemize}
    \item **Recommendations**:
        \begin{itemize}
            \item Use real-world examples to convey the societal and scientific impacts.
            \item Highlight unique aspects that position the research as indispensable.
        \end{itemize}
\end{itemize}

\subsection{Grammar, Syntax, and Formatting}
\begin{itemize}
    \item **Rating**: 6/10
    \item **Strengths**:
        \begin{itemize}
            \item Generally well-structured sentences that convey technical information.
        \end{itemize}
    \item **Weaknesses**:
        \begin{itemize}
            \item Typographical errors and alignment issues with mathematical expressions.
        \end{itemize}
    \item **Recommendations**:
        \begin{itemize}
            \item Thorough proofreading to catch any typographical errors.
            \item Ensure all mathematical formatting aligns correctly.
        \end{itemize}
\end{itemize}

\section{Summary of Ratings and Key Findings}
\begin{table}[h]
    \centering
    \begin{tabular}{|c|c|c|}
        \hline
        \textbf{Criterion} & \textbf{Rating (1-10)} & \textbf{Key Findings} \\ \hline
        Clarity & 5 & Needs simplification and consistent acronym definitions. \\ \hline
        Precision & 6 & Some vague statements need clarification. \\ \hline
        Coherence and Flow & 4 & Abrupt transitions and repetitive ideas hinder flow. \\ \hline
        Professional Tone & 7 & Mostly formal but occasional informal phraseology. \\ \hline
        Engagement and Persuasiveness & 5 & Lacks strong narrative; could benefit from examples. \\ \hline
        Grammar, Syntax, and Formatting & 6 & Generally good, but needs proofreading and formatting checks. \\ \hline
    \end{tabular}
    \caption{Summary of Assessment Ratings}
    \label{tab:summary}
\end{table}

\end{document}