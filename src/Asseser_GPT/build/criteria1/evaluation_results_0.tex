\documentclass{article}
\usepackage{geometry}
\geometry{a4paper, margin=1in}
\usepackage{amsmath}

\begin{document}

\section{Assessment of CRC Proposal for the EASE Project}

\subsection{Clarity}
\begin{itemize}
    \item \textbf{Rating:} 7/10
    \item \textbf{Strengths:}
        \begin{itemize}
            \item The proposal explains technical concepts like Bayesian networks and inference well.
            \item Important terms such as ``most probable explanation'' and ``i.i.d. assumption'' are defined.
        \end{itemize}
    \item \textbf{Weaknesses:}
        \begin{itemize}
            \item Some sections, like probabilistic graphical models, could use simpler language for an interdisciplinary audience.
            \item Extensive use of symbols without immediate textual explanation reduces readability.
        \end{itemize}
    \item \textbf{Recommendations:}
        \begin{itemize}
            \item Introduce complex terminology with more accessible definitions.
            \item Provide a glossary of symbols used in equations.
        \end{itemize}
\end{itemize}

\subsection{Precision}
\begin{itemize}
    \item \textbf{Rating:} 8/10
    \item \textbf{Strengths:}
        \begin{itemize}
            \item Methodologies like the use of Bayes' theorem are described with detailed mathematical equations.
            \item Offers specific examples of inference types (e.g., MAP and MPE).
        \end{itemize}
    \item \textbf{Weaknesses:}
        \begin{itemize}
            \item The transition from mathematical description to practical application is sometimes abrupt.
        \end{itemize}
    \item \textbf{Recommendations:}
        \begin{itemize}
            \item Bridge theoretical descriptions with more real-world scenarios to provide context.
        \end{itemize}
\end{itemize}

\subsection{Coherence and Flow}
\begin{itemize}
    \item \textbf{Rating:} 6/10
    \item \textbf{Strengths:}
        \begin{itemize}
            \item Logical progression in explaining probabilistic reasoning.
        \end{itemize}
    \item \textbf{Weaknesses:}
        \begin{itemize}
            \item Sections feel disjointed with abrupt shifts from one inference type to another.
        \end{itemize}
    \item \textbf{Recommendations:}
        \begin{itemize}
            \item Use transition statements to seamlessly connect sections.
        \end{itemize}
\end{itemize}

\subsection{Professional Tone}
\begin{itemize}
    \item \textbf{Rating:} 9/10
    \item \textbf{Strengths:}
        \begin{itemize}
            \item Maintains a formal and scholarly tone consistently.
            \item Avoids colloquial language.
        \end{itemize}
    \item \textbf{Weaknesses:}
        \begin{itemize}
            \item At times the text may underplay the project’s significance by focusing heavily on technical details.
        \end{itemize}
    \item \textbf{Recommendations:}
        \begin{itemize}
            \item Balance technical rigor with statements about the project's broader impact.
        \end{itemize}
\end{itemize}

\subsection{Engagement and Persuasiveness}
\begin{itemize}
    \item \textbf{Rating:} 7/10
    \item \textbf{Strengths:}
        \begin{itemize}
            \item Clear explanation of inference methodologies enhances persuasive power.
        \end{itemize}
    \item \textbf{Weaknesses:}
        \begin{itemize}
            \item Limited emphasis on societal impacts; focuses mostly on the technical side.
        \end{itemize}
    \item \textbf{Recommendations:}
        \begin{itemize}
            \item Integrate arguments on how this research addresses current real-world problems.
        \end{itemize}
\end{itemize}

\subsection{Grammar, Syntax, and Formatting}
\begin{itemize}
    \item \textbf{Rating:} 8/10
    \item \textbf{Strengths:}
        \begin{itemize}
            \item Generally free from grammatical errors.
            \item Equations well integrated with textual descriptions.
        \end{itemize}
    \item \textbf{Weaknesses:}
        \begin{itemize}
            \item Some sections could benefit from better typographic consistency.
        \end{itemize}
    \item \textbf{Recommendations:}
        \begin{itemize}
            \item Ensure uniform use of headings and subheadings.
        \end{itemize}
\end{itemize}

\section{Summary Table of Ratings}

\begin{center}
\begin{tabular}{|l|c|}
\hline
\textbf{Criterion} & \textbf{Rating (1-10)} \\
\hline
Clarity & 7 \\
Precision & 8 \\
Coherence and Flow & 6 \\
Professional Tone & 9 \\
Engagement and Persuasiveness & 7 \\
Grammar, Syntax, and Formatting & 8 \\
\hline
\end{tabular}
\end{center}

\end{document}