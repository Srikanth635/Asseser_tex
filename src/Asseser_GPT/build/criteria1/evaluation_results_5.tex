\documentclass{article}
\usepackage{geometry}
\usepackage{amsmath}
\usepackage{booktabs}

\geometry{a4paper, margin=1in}

\begin{document}

\title{Assessment Report for the CRC Proposal: EASE Project}
\author{}
\date{}
\maketitle

\section{Assessment}

\subsection{Clarity}
\paragraph{Rating:} 6/10
\begin{itemize}
    \item \textbf{Strengths:} The proposal uses technical language appropriate for a specialized audience.
    \item \textbf{Weaknesses:} Technical concepts and methodologies, such as Equations (2.12) and (2.13), are complex and not explained in a way that is easily accessible to an interdisciplinary audience.
    \item \textbf{Recommendations:} Provide more straightforward explanations of key equations and define acronyms consistently throughout the text.
\end{itemize}

\subsection{Precision}
\paragraph{Rating:} 7/10
\begin{itemize}
    \item \textbf{Strengths:} The proposal gives detailed descriptions of probabilistic models.
    \item \textbf{Weaknesses:} Some statements, such as the necessity and benefits of PRMs, lack detailed backing evidence and specific examples.
    \item \textbf{Recommendations:} Include more specific examples and references to support claims about the advantages of PRMs.
\end{itemize}

\subsection{Coherence and Flow}
\paragraph{Rating:} 6/10
\begin{itemize}
    \item \textbf{Strengths:} Structured with sections which guide the reader through different concepts.
    \item \textbf{Weaknesses:} Transitions between technical discussions and broader themes are abrupt, causing disjointed reading experience.
    \item \textbf{Recommendations:} Use transitional phrases to better connect distinct sections and improve overall flow.
\end{itemize}

\subsection{Professional Tone}
\paragraph{Rating:} 8/10
\begin{itemize}
    \item \textbf{Strengths:} The document maintains a formal tone appropriate for an academic proposal.
    \item \textbf{Weaknesses:} Slight tendency to use overly complex language in sections that could benefit from simplicity.
    \item \textbf{Recommendations:} Strive for balance between formality and clarity by simplifying language when possible without losing rigor.
\end{itemize}

\subsection{Engagement and Persuasiveness}
\paragraph{Rating:} 5/10
\begin{itemize}
    \item \textbf{Strengths:} Highlights the potential impact of probabilistic models to some degree.
    \item \textbf{Weaknesses:} The proposal fails to convincingly differentiate its approaches from existing solutions.
    \item \textbf{Recommendations:} Emphasize unique aspects and contributions of the research more compellingly.
\end{itemize}

\subsection{Grammar, Syntax, and Formatting}
\paragraph{Rating:} 7/10
\begin{itemize}
    \item \textbf{Strengths:} Generally grammatically correct and uses appropriate academic syntax.
    \item \textbf{Weaknesses:} Formatting issues, such as overuse of mathematical notation within the main text affecting readability.
    \item \textbf{Recommendations:} Review formatting to enhance readability, employing visuals where beneficial.
\end{itemize}

\section{Summary Ratings}

\begin{center}
\begin{tabular}{l c}
\toprule
Criterion & Rating (1-10) \\
\midrule
Clarity & 6 \\
Precision & 7 \\
Coherence and Flow & 6 \\
Professional Tone & 8 \\
Engagement and Persuasiveness & 5 \\
Grammar, Syntax, and Formatting & 7 \\
\bottomrule
\end{tabular}
\end{center}

\end{document}